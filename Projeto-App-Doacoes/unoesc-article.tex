%% Classe de documento e opções
\documentclass[%% Opções: [*] comente para remover; [>] passada para pacotes
  article,%% Tipo de documento: article, book, report, etc. [>]
  a4paper,%% Tamanho de papel: a4paper, letterpaper, etc. [>]
  12pt,%% Tamanho de fonte: 10pt, 11pt, 12pt, etc. [>]
  fleqn,%% Alinhamento de equações à esquerda (comente para centralizado) [>]
  oneside,%% Impressão: oneside (anverso) ou twoside (anverso e verso) [>]
  % twocolumn,%% Texto em duas colunas (comente para uma coluna) [>]
  chapter = TITLE,%% Títulos de capítulos em maiúsculas [*]
  section = TITLE,%% Títulos de seções (secundárias) em maiúsculas [*]
]{abntex2}

%% Pacotes utilizados
\usepackage[%% Opções
  BibURLs = false,%% Links de URLs nas referências: true ou false
  ABNTNum = none,%% Estilo numérico ABNT: none (AUTOR, ANO), dflt (1) e brkt [1]
]{unoesc-article}

\usepackage{caption}




%% Arquivo de referências
\addbibresource{unoesc-article.bib}

%% Informações do documento
%%%% Título
\titulo{PROJETO de APLICAÇÃO PARA CENTRALIZAÇÃO DE DOAÇÕES}
%%%% Autor(es) e afiliação(ões)



\autor{%
  Fabrinni Ferreira; \\
  \\
  Pedro Thomas; \\
  \\
  Vinicius Leban;%
  \and Prof. Jacson Luiz Matte %
  \thanks{%
    \affil{Universidade do Oeste Catarinense; Sistemas de Informação; Disciplina: Prática Profissional e Inserção Comunitária I}%
    \sep\email{jacson.matte@unoesc.edu.br}%
  }%
}

\data{}
%% Inserir imagens
\usepackage{graphicx}

%% Ferramenta para criação de índices
\makeindex%
\crefname{figure}{Figura}{Figuras}
\crefname{table}{Quadro}{Quadros}
%% Início do documento
\begin{document}

\pretextual%% Elementos pré-textuais

\begin{paginadetitulo}%% Página de título
  
\end{paginadetitulo}

\textual%% Elementos textuais
\newpage
\section{Introdução}

A solidariedade e a cooperação social são fatores fundamentais para reduzir desigualdades e melhorar a qualidade de vida em comunidades urbanas. Entretanto, muitas pessoas que desejam ajudar encontram barreiras para localizar instituições confiáveis, entender quais itens são mais necessários ou garantir que sua doação seja bem aproveitada. Por outro lado, organizações não governamentais (ONGs), projetos sociais e famílias em situação de vulnerabilidade enfrentam dificuldades para divulgar suas demandas e manter um fluxo contínuo de arrecadações.

Nesse contexto, a tecnologia surge como uma ferramenta estratégica para aproximar quem pode doar de quem necessita de apoio. Plataformas digitais e aplicativos móveis permitem criar redes de colaboração, oferecendo mais agilidade, segurança e transparência no processo de doação. Assim, este projeto propõe o desenvolvimento de um aplicativo de rede social solidária, que conecte pessoas, ONGs e outros parceiros, permitindo a publicação de itens para doação, campanhas de arrecadação e validação das entregas de maneira simples e interativa. A proposta busca tornar o ato de doar mais acessível, estimulando a participação da sociedade e fortalecendo uma cultura de ajuda mútua.


\section{DELIMITAÇÃO DO TEMA E JUSTIFICATIVA}

O presente estudo tem como foco a criação de uma plataforma digital para intermediação de doações, permitindo que usuários – sejam pessoas físicas ou organizações – possam cadastrar perfis, divulgar itens disponíveis, solicitar doações, criar campanhas e acompanhar o histórico de suas contribuições. O sistema funcionará de maneira semelhante a uma rede social: cada perfil poderá publicar postagens, interagir com outros usuários e validar transações.

Entre as funcionalidades planejadas, destacam-se:
\begin{itemize}
    \item Cadastro de usuários (doadores, beneficiários e ONGs);
    \item Publicações de doação e campanhas em formato de feed interativo;
    \item Histórico de doações para acompanhamento do impacto gerado;
    \item Sistema de gamificação em que doadores e ONGs acumulam pontos a cada ação validada;
    \item Futuras parcerias com prefeituras e empresas, permitindo que a gamificação gere benefícios, como descontos em taxas municipais ou incentivos semelhantes aos programas de doação de sangue.
\end{itemize}

A escolha do tema é justificada por sua relevância social e potencial de impacto comunitário. Em um cenário de desigualdade econômica, muitas instituições dependem de doações, mas enfrentam dificuldades em manter canais eficientes de comunicação. A plataforma proposta busca integrar tecnologia e responsabilidade social, oferecendo um ambiente seguro, transparente e escalável, que incentive a prática da solidariedade.

\section{OBJETIVO GERAL}

Desenvolver um aplicativo multiplataforma que funcione como uma rede social de doações, conectando pessoas, ONGs e instituições públicas, permitindo a publicação de itens, a organização de campanhas e a validação das entregas. O sistema deverá contar com recursos de gamificação, geolocalização e histórico de doações, visando aumentar o engajamento dos usuários e fomentar uma rede de solidariedade contínua. A solução também pretende criar uma base de dados que permita futuras parcerias com órgãos governamentais e empresas privadas, abrindo espaço para políticas de incentivo fiscal e programas de reconhecimento social que beneficiem tanto doadores quanto instituições participantes.

\section{PÚBLICO ALVO}

O público-alvo do aplicativo compreende dois grupos principais. O primeiro são os \textbf{doadores}, que incluem pessoas físicas, empresas e instituições interessadas em contribuir com roupas, alimentos e outros itens essenciais. O segundo grupo são os \textbf{beneficiados}, formado por famílias e indivíduos em situação de vulnerabilidade social na comunidade de Chapecó-SC. 
Além disso, organizações não governamentais e entidades assistenciais locais também podem se beneficiar da plataforma, utilizando-a como apoio na gestão das doações e na distribuição de recursos. Dessa forma, o aplicativo busca integrar diferentes atores sociais em prol da solidariedade e do fortalecimento comunitário.

\section{TRABALHOS RELACIONADOS}

A análise de trabalhos relacionados posiciona o projeto, destacando seus diferenciais frente a soluções existentes de solidariedade.

\begin{itemize}
    \item \textbf{Plataformas Logísticas Formais (Ex: Exército da Salvação, Doações.gov.br, Rede Caixa Solidária):} Oferecem canais formais e garantem o destino dos itens, mas dependem de \textbf{logística centralizada e pontos de coleta físicos}. O diferencial do projeto é a \textbf{intermediação direta (peer-to-peer) via aplicativo móvel}, com \textbf{feed social} para maior agilidade e engajamento.
    \item \textbf{Canais Digitais Informais (Ex: Solidarizando, Grupos de Redes Sociais):} São ágeis na publicação, mas carecem de \textbf{segurança, validação formal da entrega e histórico centralizado}. O projeto se diferencia ao fornecer uma \textbf{rede social estruturada e transparente}, com validação mútua da entrega e registro do impacto (RF04 e RF05).
    \item \textbf{Sistemas com Incentivos (Ex: Nota Fiscal Paulista):} Demonstram o potencial de vincular ações sociais a benefícios tangíveis (incentivo fiscal/social). O aplicativo utiliza este conceito de forma nativa, com \textbf{gamificação por pontos} (RF07) para futura conversão em benefícios, focando o incentivo na \textbf{base da transação de doação de itens}.
\end{itemize}

O projeto busca ser a solução que integra a transparência de um sistema formal, a interação de uma rede social e os incentivos de programas governamentais, focando na comunidade de Chapecó-SC.
\item % \textbf{riscos}:
\end{itemize}


Um pouco mais sobre as aplicações relacionadas que foram citadas a cima:
\item % \textbf{riscos}:
\end{itemize}

{\textbf{Exército da Salvação}}

\begin{itemize}
    \item É uma das maiores instituições mundiais de doação de roupas, móveis e itens usados.
    \item Usa caminhões para coletar itens nas casas das pessoas.
    \item Depende de centros regionais e lojas beneficentes.
    \item Possui credibilidade, mas a coleta pode ser \textbf{lenta}, e nem sempre disponível em todas as regiões.
\end{itemize}

{\textbf{Doações.gov.br}}

\begin{itemize}
    \item Plataforma oficial do governo brasileiro.
    \item Atua sobretudo em \textbf{doações institucionais}, ligando empresas e entidades.
    \item Foca em grandes volumes e demandas de órgãos públicos.
    \item Não tem foco em pessoa comum doando para outra pessoa comum.
\end{itemize}

{\textbf{Rede Caixa Solidária}}

\begin{itemize}
    \item Conjunto de pontos de coleta mantidos por instituições parceiras.
    \item Presente em vários estados.
    \item Funciona muito bem para doação de roupas e alimentos.
    \item O processo depende de deslocamento físico, o que reduz acessibilidade para comunidades carentes.
\end{itemize}


{\textbf{Solidarizando}}

\begin{itemize}
    \item Plataforma digital com foco em postagens de pedidos e doações.
    \item A interação é simples, mas ainda funciona com base em \textbf{confiança sem verificação}.
    \item Falta controle de histórico, indicadores, certificação da entrega, etc.
\end{itemize}

{\textbf{Grupos de redes sociais}}

\begin{itemize}
    \item Milhares de grupos de Facebook e WhatsApp de doações locais.
    \item Funcionam como classificados: alguém posta → alguém comenta → entrega é combinada.
    \item Muito rápidos e com grande alcance.
\end{itemize}
 


\section{REQUISITOS FUNCIONAIS (RF)}

Os Requisitos Funcionais (RFs) definem claramente as funcionalidades que o aplicativo deve fornecer para atender às necessidades dos usuários envolvidos (Doadores, Beneficiários e ONGs):

\begin{itemize}
    \item \textbf{RF01 — Gerenciamento de Perfis e Autenticação:} O sistema deve permitir o cadastro, login e atualização de dados dos usuários, incluindo nome, e-mail, localização, tipo de perfil e pontuação.

    \item \textbf{RF02 — Publicação de Itens e Campanhas:} O sistema deve permitir que Doadores e ONGs publiquem itens de doação (com categoria) ou campanhas (com meta e valor arrecadado), contendo título, descrição, data e status (Disponível / Solicitado / Entregue).

    \item \textbf{RF03 — Solicitação e Interação da Doação:} O sistema deve permitir que Beneficiários visualizem publicações próximas via geolocalização, solicitem itens/campanhas e interajam por chat para combinar a entrega.

    \item \textbf{RF04 — Validação Mútua da Entrega:} Após a entrega, o Doador deve confirmar a entrega e o Beneficiário deve confirmar o recebimento; o sistema deve registrar a transação como VALIDADA quando ambas as confirmações ocorrerem.

    \item \textbf{RF05 — Sistema de Pontuação (Gamificação):} Quando uma transação for validada, o sistema deve calcular e atribuir pontos aos perfis envolvidos, atualizar a pontuação e registrar o histórico.

    \item \textbf{RF06 — Registro Completo de Transações:} O sistema deve registrar datas de solicitação, entrega, status e vínculo entre doador e beneficiário.

    \item \textbf{RF07 — Sistema de Geolocalização:} O sistema deve registrar a localização do usuário e exibir publicações no feed de acordo com a proximidade.

    \item \textbf{RF08 — Feed de Publicações:} O sistema deve apresentar um feed filtrável por proximidade, categoria (itens) e tipo (item ou campanha).

    \item \textbf{RF09 — Histórico de Doações:} O sistema deve exibir o histórico de transações realizadas por cada perfil, contendo registros das validações.

    \item \textbf{RF10 — Integração com Parceiros Externos (Objetivo Futuro):} O sistema deve possibilitar a geração de uma base exportável para parceiros externos, como órgãos de assistência social.
\end{itemize}

\item % \textbf{riscos}:
\end{itemize}
\vspace{2cm}

\section{REQUISITOS NÃO FUNCIONAIS (RNF)}

Os Requisitos Não Funcionais (RNFs) descrevem critérios de qualidade, segurança, desempenho e tecnologia necessários para garantir a eficiência do sistema:

\begin{itemize}
    \item \textbf{RNF01 — Multiplataforma:} O aplicativo deve funcionar em dispositivos Android e iOS.

    \item \textbf{RNF02 — Integração via API REST:} Toda comunicação entre o aplicativo e o backend deve ocorrer via API REST padronizada. 

    \item \textbf{RNF03 — Segurança e Autenticação:} O sistema deve garantir proteção de dados pessoais, autenticação segura e evitar alterações não autorizadas de perfis e transações.

    \item \textbf{RNF04 — Disponibilidade e Confiabilidade:} O sistema deve permanecer disponível de forma contínua, garantindo consistência e registro correto das transações.

    \item \textbf{RNF05 — Desempenho:} Operações essenciais como login, publicação, solicitação e validação devem responder em até 2 segundos, em média.

    \item \textbf{RNF06 — Usabilidade:} A interface deve ser intuitiva, facilitando publicações, solicitações, validações e visualização de pontos por usuários leigos. 

    \item \textbf{RNF07 — Geolocalização Precisa:} O sistema deve utilizar serviços de localização confiáveis, atualizando a posição do usuário em tempo real.

    \item \textbf{RNF08 — Persistência de Dados:} Todas as informações devem ser armazenadas em banco de dados SQL ou NoSQL conforme a necessidade dos serviços. 
\end{itemize}
\vspace{0.2cm}

\section{DIAGRAMAS}

Os diagramas são fundamentais na modelagem de sistemas, pois permitem visualizar de forma clara estruturas, fluxos e interações que seriam difíceis de compreender apenas por texto. Eles facilitam a comunicação entre os envolvidos, reduzem ambiguidades, ajudam na validação dos requisitos e permitem identificar problemas antes da implementação. Assim, tornam o desenvolvimento mais organizado, preciso e eficiente, contribuindo para a qualidade do sistema.
\item % \textbf{riscos}:
\end{itemize}
Sobre os diagramas, iremos apresentar os seguintes diagramas da nossa aplicação a baixo: 

\begin{itemize}
    \item \textbf{Diagrama de Casos de Uso}
    \item \textbf{Diagrama de Classes}
    \item \textbf{Diagramas de Sequência}
    \item \textbf{Diagrama de Atividades}
    \item \textbf{Diagrama de Componentes}
\end{itemize}

%% DIAGRAMA DE CASO DE USO
\subsubsection{Diagrama de Caso de Uso}

O diagrama apresenta os principais atores do sistema — doador, beneficiário, ONG e parceiro, e mostra as funcionalidades que cada um pode executar, como publicar itens, solicitar apoio, interagir via chat, buscar por geolocalização, validar entregas e visualizar dados de impacto. Também evidencia relações entre casos de uso, como extensões e inclusões, indicando quando uma funcionalidade depende ou complementa outra. No geral, o diagrama organiza de forma clara como cada ator interage com o sistema e quais ações estão disponíveis para cada perfil.

\begin{center}
    \includegraphics[width=0.85\textwidth]{Projeto-App-Doacoes/images/casodeuso.jpeg}\\
    \vspace{0.2cm}
    \textbf{Figura 1 — Diagrama de Caso de Uso do Sistema}
\end{center}
\vspace{1cm}
%% DIAGRAMA DE CLASSES
\subsubsection{Diagrama de Classes}

 O diagrama de classes apresenta a estrutura estática do sistema, mostrando as principais entidades, seus atributos, métodos e como elas se relacionam. As classes derivadas de Usuário — como Doador, Beneficiário e ONG — aparecem especializadas com funcionalidades próprias, enquanto elementos centrais como ItemDoacao, Campanha e Transacao mostram o fluxo de publicação, contribuição e registro das operações. O diagrama também evidencia associações importantes, multiplicidades e classes auxiliares, como Chat e HistoricoImpacto, organizando de forma clara como os componentes do sistema se conectam e sustentam as funcionalidades previstas.
\begin{center}
    \includegraphics[width=0.95\textwidth]{Projeto-App-Doacoes/images/diagramadeclasses.jpeg}\\
    \vspace{0.2cm}
    \textbf{Figura 2 — Diagrama de Classes do Sistema}
\end{center}
%% DIAGRAMA DE SEQUENCIA
\subsubsection{Diagrama de Sequencia}

O diagrama de sequência descreve a ordem das interações entre o Doador, o Beneficiário, o aplicativo, a API e o módulo de histórico durante a confirmação da transação. Ele mostra como as mensagens são trocadas ao longo do processo, incluindo a solicitação de confirmação, as validações feitas, a atualização do status da doação e o registro do evento, evidenciando o fluxo temporal da operação.
\begin{center}
    \includegraphics[width=0.74\textwidth]{Projeto-App-Doacoes/images/diagramadesequencia.jpeg}\\
    \vspace{0.2cm}
    \textbf{Figura 3 — Diagrama de sequência}
\end{center}
%% DIAGRAMA DE ATIVIDADES
\subsubsection{Diagrama de Atividades}

O diagrama mostra o fluxo de confirmação de uma entrega: doador e recebedor acessam a transação e confirmações, enquanto o sistema atualiza o status para indicar quem já confirmou. Em seguida, o sistema verifica se ambos concluíram o processo. Se sim, a transação é validada, os pontos são calculados e adicionados, e um registro é criado no histórico. Se não, o processo permanece pendente.
\begin{center}
    \includegraphics[width=0.63\textwidth]{Projeto-App-Doacoes/images/diagramadeatividade.jpeg}\\
    \vspace{0.2cm}
    \textbf{Figura 4 — Diagrama de Atividades}
\end{center}
%% DIAGRAMA DE COMPONENTES
\subsubsection{Diagrama de Componentes}

O diagrama mostra o funcionamento interno do app: o usuário acessa o app, que envia requisições para a API. A API repassa ações para módulos de usuários, doações, transações, chat e gamificação, enquanto tudo é salvo no banco de dados. Um serviço de mapas fornece geolocalização para filtrar itens por proximidade. Assim, o sistema integra cadastro, publicações, comunicação e registro de doações.
\begin{center}
    \includegraphics[width=0.85\textwidth]{Projeto-App-Doacoes/images/diagramadecomponentes.jpeg}\\
    \vspace{0.2cm}
    \textbf{Figura 5 — Diagrama de Componentes}
\end{center}


\postextual
\newpage
\printbibliography

\end{document}